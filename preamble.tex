\usepackage[utf8]{inputenc}
\usepackage[a4paper]{geometry}
\usepackage[english]{babel}
\usepackage{cancel}
\usepackage{quotchap}
\usepackage{amsmath}
\usepackage{amssymb}
\usepackage{amsthm}
\usepackage{appendix}
\usepackage{array}
\usepackage{marvosym}
\usepackage[hidelinks]{hyperref}
\usepackage{color}
\usepackage[table,dvipsnames]{xcolor}
\usepackage{amsthm}
\usepackage{graphicx}
\usepackage{polynom}
\usepackage{braket}
\usepackage{fontawesome}
\usepackage{mathtools}
\usepackage{tikz}
\usepackage{tkz-euclide}
\usepackage[most]{tcolorbox}
\usepackage{pgfplots}
\usepackage{multicol}
\usepgfplotslibrary{polar}
\usepgflibrary{shapes.geometric}
\usetikzlibrary{calc}
\pgfplotsset{def/.append style={axis x line=middle, axis y line=middle, xlabel={\(x\)}, ylabel={\(y\)}, axis equal}}
\pgfplotsset{compat=1.17}
\graphicspath{{./images/}}
\usetikzlibrary{fit,shapes}
\usetikzlibrary{graphs}

\DeclareMathOperator{\arcsec}{arcsec}
\DeclareMathOperator{\arccot}{arccot}
\DeclareMathOperator{\arccsc}{arccsc}
\DeclarePairedDelimiter{\ceil}{\lceil}{\rceil}
\DeclarePairedDelimiter{\floor}{\lfloor}{\rfloor}

\usepackage{fix-cm} 
\makeatletter
\newcommand\HUGE{\@setfontsize\Huge{50}{60}} 
\makeatother

\newcommand{\subsubsubsection}[1]{\paragraph{#1}\mbox{}\\}
\newcommand*{\?}{\stackrel{?}{=}}
\newcommand*{\dd}{\,\text{d}}
\newcommand*{\thus}{.^..}
%\renewcommand*{\and}{\wedge}
%\renewcommand*{\or}{\vee}
\newcommand*{\curl}{\text{curl\,}}
\newcommand*{\range}{\text{range\,}}
\renewcommand*{\div}{\text{div\,}}
\newcommand*{\Difficulty}{\Radioactivity}
%\newcommand*{\Difficulty}{\Frowny}
\newcommand*{\Stop}{\Stopsign}
\newcommand*{\DOTHISLATER}{\begin{center}\Stop\Stop\Stop\Stop\Stop\Stop\Stop\Stop\Stop\Stop\Stop\Stop\Stop\Stop\Stop\end{center}}
\newcommand*{\laplace}[2][]{\mathcal{L}^{#1}\left[#2\right]}
\newcommand*{\powerset}[1]{\mathcal{P}\left(#1\right)}
\renewcommand*{\vec}[1]{\overset{_{\rightharpoonup}}{#1}}
\renewcommand*{\bar}[1]{\overline{#1}}
%\renewcommand*{\vec}[1]{\mathbf{#1}}
\newcommand*{\nin}{\notin}
\newcommand*{\nequiv}{\not\equiv}
\newcommand*{\proj}{\text{proj\,}}
\newcommand*{\rank}{\text{rank\,}}
\newcommand*{\nullity}{\text{nullity\,}}
\newcommand*{\trace}{\text{trace\,}}
\newcommand*{\iprod}[2]{\left\langle #1,#2\right\rangle}
\newcommand*{\rref}{\text{rref\,}}
\newcommand*{\ef}{\text{ref\,}}
\newcommand*{\interior}{\text{int\,}}
\newcommand*{\boundary}{\text{bd\,}}
\newcommand*{\cl}{\text{cl\,}}
\DeclareMathOperator{\lcm}{lcm}
\DeclareMathOperator{\Sym}{Sym}
\DeclareMathOperator{\GL}{GL}
\DeclareMathOperator{\SL}{SL}
\DeclareMathOperator{\stab}{stab}
\DeclareMathOperator{\wstab}{wstab}
\DeclareMathOperator{\Span}{span}

\usepackage{sfmath}
\renewcommand{\familydefault}{\sfdefault}

% \renewcommand*{\qedsymbol}{\(\blacksquare\)}
\newtheorem*{remark*}{Remark}
\newtheorem*{question*}{Question}
\newtheorem*{answer*}{Answer}
\newtheorem{example}{Example}
% % \newtheorem*{corollary*}{Corollary}
% \newtcbtheorem[number within=section]{definition}{Definition}{
%                 lower separated=false,
%                 breakable,
%                 before skip=0.5cm,
%                 colback=white,
%                 colframe=black,fonttitle=\bfseries,
%                 colbacktitle=black,
%                 coltitle=white,
%                 enhanced,
%                 attach boxed title to top left={yshift=-0.1in,xshift=0.15in},
%                 }{def}
                
% \newtcbtheorem[number within=section]{example}{Example}{
%                 lower separated=false,
%                 breakable,
%                 before skip=0.5cm,
%                 colback=gray!15,
%                 colframe=white!10!black,fonttitle=\bfseries,
%                 colbacktitle=black,
%                 coltitle=white,
%                 enhanced,
%                 attach boxed title to top left={yshift=-0.1in,xshift=0.15in},
%                 }{exa}
                
% \newtcbtheorem[number within=section]{exercise}{Exercise}{
%                 lower separated=false,
%                 breakable,
%                 before skip=0.5cm,
%                 colback=gray!15,
%                 colframe=white!10!black,fonttitle=\bfseries,
%                 colbacktitle=black,
%                 coltitle=white,
%                 enhanced,
%                 attach boxed title to top left={yshift=-0.1in,xshift=0.15in},
%                 }{exe}

% \newtcbtheorem[number within=section]{lemma}{Lemma}{
%                 lower separated=false,
%                 breakable,
%                 before skip=0.5cm,
%                 colback=white,
%                 colframe=black,fonttitle=\bfseries,
%                 colbacktitle=black,
%                 coltitle=white,
%                 enhanced,
%                 attach boxed title to top left={yshift=-0.1in,xshift=0.15in},
%                 }{lem}

% \newtcbtheorem[number within=section]{proposition}{Proposition}{
%                 lower separated=false,
%                 breakable,
%                 before skip=0.5cm,
%                 colback=white,
%                 colframe=black,fonttitle=\bfseries,
%                 colbacktitle=black,
%                 coltitle=white,
%                 enhanced,
%                 attach boxed title to top left={yshift=-0.1in,xshift=0.15in},
%                 }{prop}

% \newtcbtheorem[number within=section]{theorem}{Theorem}{
%                 lower separated=false,
%                 breakable,
%                 before skip=0.5cm,
%                 colback=white,
%                 colframe=black,fonttitle=\bfseries,
%                 colbacktitle=black,
%                 coltitle=white,
%                 enhanced,
%                 attach boxed title to top left={yshift=-0.1in,xshift=0.15in},
%                 }{thm}
                
% \newtcbtheorem[number within=section]{corollary}{Corollary}{
%     lower separated=false,
%     breakable,
%     before skip=0.5cm,
%     colback=white,
%     colframe=black,fonttitle=\bfseries,
%     colbacktitle=black,
%     coltitle=white,
%     enhanced,
%     attach boxed title to top left={yshift=-0.1in,xshift=0.15in},
%     }{cor}
                    

%% COLORED BOXES
\newtcbtheorem[number within=section]{solution}{Solution}{
                lower separated=false,
                before skip=0.5cm,
                breakable,
                colback=white,
                colframe=black,fonttitle=\bfseries,
                colbacktitle=black,
                coltitle=white,
                enhanced,
                attach boxed title to top left={yshift=-0.1in,xshift=0.15in},
                }{sol}

\newtcbtheorem[number within=section]{definition}{Definition}{
                lower separated=false,
                breakable,
                before skip=0.5cm,
                colback=white,
                colframe=violet,fonttitle=\bfseries,
                colbacktitle=violet,
                coltitle=white,
                enhanced,
                attach boxed title to top left={yshift=-0.1in,xshift=0.15in},}{def}
                    
\newtcbtheorem[number within=section]{exercise}{Exercise}{
                lower separated=false,
                breakable,
                before skip=0.5cm,
                colback=white,
                colframe=violet,fonttitle=\bfseries,
                colbacktitle=violet,
                coltitle=white,
                enhanced,
                attach boxed title to top left={yshift=-0.1in,xshift=0.15in},
                }{exe}
                
\newtcbtheorem[number within=section]{theorem}{Theorem}{
                lower separated=false,
                breakable,
                before skip=0.5cm,
                colback=white,
                colframe=Blue,fonttitle=\bfseries,
                colbacktitle=Blue,
                coltitle=white,
                enhanced,
                attach boxed title to top left={yshift=-0.1in,xshift=0.15in},
                }{thm}
                
\newtcbtheorem[number within=section]{corollary}{Corollary}{
                lower separated=false,
                breakable,
                before skip=0.5cm,
                colback=white,
                colframe=Blue,fonttitle=\bfseries,
                colbacktitle=Blue,
                coltitle=white,
                enhanced,
                attach boxed title to top left={yshift=-0.1in,xshift=0.15in},
                }{cor}

\newtcbtheorem[number within=section]{proposition}{Proposition}{
                lower separated=false,
                breakable,
                before skip=0.5cm,
                colback=white,
                colframe=Blue,fonttitle=\bfseries,
                colbacktitle=Blue,
                coltitle=white,
                enhanced,
                attach boxed title to top left={yshift=-0.1in,xshift=0.15in},
                }{prop}

\newtcbtheorem[number within=section]{lemma}{Lemma}{
                lower separated=false,
                breakable,
                before skip=0.5cm,
                colback=white,
                colframe=Blue,fonttitle=\bfseries,
                colbacktitle=Blue,
                coltitle=white,
                enhanced,
                attach boxed title to top left={yshift=-0.1in,xshift=0.15in},
                }{lem}

\usepackage{listings}


% Configuration for Syntax Highlighting
\definecolor{codegreen}{rgb}{0,0.6,0}
\definecolor{codegray}{rgb}{0.5,0.5,0.5}
\definecolor{codepurple}{rgb}{0.58,0,0.82}
\definecolor{backcolour}{rgb}{0.95,0.95,0.92}
\definecolor{bggray}{gray}{0.9}
\definecolor{cadmiumgreen}{rgb}{0.0, 0.42, 0.24}
\definecolor{violet}{rgb}{0.32, 0.0, 0.65}


\renewcommand{\ttdefault}{pcr}
\lstdefinestyle{code}{
    backgroundcolor=\color{bggray},   
    commentstyle=\color{gray},
    keywordstyle=\bfseries,
    morekeywords={def},
    numberstyle=\tiny\color{black},
    %stringstyle=\color{codepurple},
    basicstyle=\ttfamily\footnotesize,
    breakatwhitespace=false,         
    breaklines=true,                 
    captionpos=b,                    
    keepspaces=true,                 
    numbers=left,                    
    numbersep=5pt,                  
    showspaces=false,                
    showstringspaces=false,
    showtabs=false,                  
    tabsize=2
}

\lstset{style=code}

\newenvironment{code}{\fontfamily{lmtt}\selectfont}{\par}