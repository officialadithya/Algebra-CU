
\begin{multicols}{2}
      \setlength{\parindent}{0pt}
      \footnotesize{

\textsc{Definition} \ref{def:functions}, \textsc{Page} \pageref{def:functions} \textit{Functions} \\
\textsc{Definition} \ref{def:groups}, \textsc{Page} \pageref{def:groups} \textit{Groups} \\
\textsc{Proposition} \ref{prop:identityunique}, \textsc{Page} \pageref{prop:identityunique} \textit{The Group Identity is Unique} \\
\textsc{Proposition} \ref{prop:identityunique}, \textsc{Page} \pageref{prop:identityunique} \textit{The Inverse of a Group Element is Unique} \\
\textsc{Proposition} \ref{prop:inverseofproduct}, \textsc{Page} \pageref{prop:inverseofproduct} \textit{The Inverse of a Product} \\
\textsc{Proposition} \ref{prop:inverseofmanyproducts}, \textsc{Page} \pageref{prop:inverseofmanyproducts} \textit{The Inverse of Many Products} \\
\textsc{Definition} \ref{def:symmetricgroup}, \textsc{Page} \pageref{def:symmetricgroup} \textit{The Symmetric Group} \\
\textsc{Definition} \ref{def:cyclenotation}, \textsc{Page} \pageref{def:cyclenotation} \textit{Cycles} \\
\textsc{Theorem} \ref{thm:orderpermutationlcm}, \textsc{Page} \pageref{thm:orderpermutationlcm} \textit{The Order of \(\sigma \in S_n\)} \\
\textsc{Definition} \ref{def:subgroup}, \textsc{Page} \pageref{def:subgroup} \textit{Subgroups} \\
\textsc{Theorem} \ref{thm:subgroupcriterion}, \textsc{Page} \pageref{thm:subgroupcriterion} \textit{Subgroup Criterion} \\
\textsc{Definition} \ref{def:generatedsubgroup}, \textsc{Page} \pageref{def:generatedsubgroup} \textit{Generated Subgroups} \\
\textsc{Definition} \ref{def:ordergroupelement}, \textsc{Page} \pageref{def:ordergroupelement} \textit{Order of a Group Element} \\
\textsc{Definition} \ref{def:cyclicgroup}, \textsc{Page} \pageref{def:cyclicgroup} \textit{Cyclic Groups} \\
\textsc{Definition} \ref{def:ordergroup}, \textsc{Page} \pageref{def:ordergroup} \textit{Order of a Group} \\
\textsc{Definition} \ref{def:dihedralgroup}, \textsc{Page} \pageref{def:dihedralgroup} \textit{Dihedral Groups} \\
\textsc{Proposition} \ref{prop:orderofdihedralgroup}, \textsc{Page} \pageref{prop:orderofdihedralgroup} \textit{Order of the Dihedral Group} \\
\textsc{Definition} \ref{def:grouphomomorphisms}, \textsc{Page} \pageref{def:grouphomomorphisms} \textit{Group Homomorphisms} \\
\textsc{Proposition} \ref{prop:propertiesgrouphomomorphisms}, \textsc{Page} \pageref{prop:propertiesgrouphomomorphisms} \textit{Properties of Group Homomorphisms} \\
\textsc{Definition} \ref{def:grouphomomorphisms}, \textsc{Page} \pageref{def:grouphomomorphisms} \textit{Group Isomorphism} \\
\textsc{Theorem} \ref{thm:isomorphismcyclic}, \textsc{Page} \pageref{thm:isomorphismcyclic} \textit{Isomorphisms of Cyclic Groups} \\
\textsc{Theorem} \ref{thm:isomorphismsymmetric}, \textsc{Page} \pageref{thm:isomorphismsymmetric} \textit{Isomorphisms of Symmetric Groups} \\
\textsc{Definition} \ref{def:groupactions}, \textsc{Page} \pageref{def:groupactions} \textit{Group Actions} \\
\textsc{Proposition} \ref{prop:leftactioniffgroupaction}, \textsc{Page} \pageref{prop:leftactioniffgroupaction} \textit{Left Action If and Only If Group Acts on the Set} \\
\textsc{Definition} \ref{def:trivialityfaithfulnessgroupaction}, \textsc{Page} \pageref{def:trivialityfaithfulnessgroupaction} \textit{Triviality and Faithfulness} \\
\textsc{Lemma} \ref{lem:injhomoifftrivialker}, \textsc{Page} \pageref{lem:injhomoifftrivialker} \textit{Injective Group Homomorphism If and Only if Kernel is Trivial} \\
\textsc{Theorem} \ref{thm:faithfulactioniffgisotosubofSA}, \textsc{Page} \pageref{thm:faithfulactioniffgisotosubofSA} \textit{Faithful Action If and Only If \(G\) Isomorphic to Subgroup of \(S_A\)} \\
\textsc{Definition} \ref{def:abelian}, \textsc{Page} \pageref{def:abelian} \textit{Abelian Groups} \\
\textsc{Proposition} \ref{prop:abelianpreserveiso}, \textsc{Page} \pageref{prop:abelianpreserveiso} \textit{Isomorphic Groups are Either Both Abelian or Both Not Abelian} \\
\textsc{Definition} \ref{def:fields}, \textsc{Page} \pageref{def:fields} \textit{Fields} \\
\textsc{Definition} \ref{def:Fmodules}, \textsc{Page} \pageref{def:Fmodules} \textit{\(\mathbb {F}\)-Modules} \\
\textsc{Definition} \ref{def:groupactionsFmodules}, \textsc{Page} \pageref{def:groupactionsFmodules} \textit{Group Actions on \(\mathbb {F}\)-Modules} \\
\textsc{Proposition} \ref{prop:generatedsubgroupcharacterization}, \textsc{Page} \pageref{prop:generatedsubgroupcharacterization} \textit{Equivalent Characterizations of the Generated Subgroup} \\
\textsc{Definition} \ref{def:maximalsubgroup}, \textsc{Page} \pageref{def:maximalsubgroup} \textit{Maximal Subgroup} \\
\textsc{Definition} \ref{def:subgrouplattice}, \textsc{Page} \pageref{def:subgrouplattice} \textit{Subgroup Lattice} \\
\textsc{Definition} \ref{def:hassediagrams}, \textsc{Page} \pageref{def:hassediagrams} \textit{Hasse Diagrams} \\
\textsc{Definition} \ref{def:stablizers}, \textsc{Page} \pageref{def:stablizers} \textit{Stabilizers} \\
\textsc{Definition} \ref{def:centralizers}, \textsc{Page} \pageref{def:centralizers} \textit{Centralizers} \\
\textsc{Definition} \ref{def:normalizers}, \textsc{Page} \pageref{def:normalizers} \textit{Normalizers} \\
\textsc{Definition} \ref{def:centers}, \textsc{Page} \pageref{def:centers} \textit{Centers} \\
\textsc{Proposition} \ref{prop:trivialityfaithfulnessconj}, \textsc{Page} \pageref{prop:trivialityfaithfulnessconj} \textit{Conditions for Triviality and Faithfulness of Conjugation} \\
\textsc{Definition} \ref{def:normalsubgroups}, \textsc{Page} \pageref{def:normalsubgroups} \textit{Normal Subgroups} \\
\textsc{Definition} \ref{def:transitivegroupactions}, \textsc{Page} \pageref{def:transitivegroupactions} \textit{Transitivity of Group Actions} \\
\textsc{Definition} \ref{def:orbits}, \textsc{Page} \pageref{def:orbits} \textit{Orbits} \\

      }
\end{multicols}

